\documentclass{article}

\usepackage{cmap}  % should be before fontenc
\usepackage[T2A]{fontenc}
\usepackage[utf8]{inputenc}
\usepackage[russian]{babel}
\usepackage{amsmath,amssymb,amsthm}
\usepackage[pdftex,colorlinks=true,linkcolor=blue,urlcolor=red,unicode=true,hyperfootnotes=false,bookmarksnumbered]{hyperref}
\usepackage{indentfirst}

% так можно определять команду для повторяющихся обозначений,
% чтобы не набирать каждый раз заново
\newcommand{\E}{\ensuremath{\mathsf{E}}}  % матожидание
\newcommand{\D}{\ensuremath{\mathsf{D}}}  % дисперсия
\newcommand{\Prb}{\ensuremath{\mathsf{P}}}  % вероятностная мера

\newcommand{\eps}{\varepsilon}  % нормальная буква эпсилон
\renewcommand{\phi}{\varphi}  % нормальная буква фи

\renewcommand{\le}{\leqslant}  % нормальный знак <=
\renewcommand{\leq}{\leqslant}  % нормальный знак <=
\renewcommand{\ge}{\geqslant}  % нормальный знак >=
\renewcommand{\geq}{\geqslant}  % нормальный знак >=

\newtheorem{lemma}{Лемма}  % создаёт команд для лемм, можно сделать так же для любого другого вида утверждений

\pagestyle{myheadings}
\markright{Дмитрий Кошман\hfill}  % <- здесь нужно подставить свои имя и фамилию

\begin{document}

\section{Задача 2-1}

Будем вместе с $d_i$ - минимальным конечным элементом возрастающей подпоследовательности длины $i$ в текущем префиксе хранить индекс этого элемента в последовательности. Также заведем массив $L$ длины $n$, такой, что $L_i$ равен индексу предыдущего элемента в возрастающей подпоследовательности максимальной длины, в которую входит элемент с индексом $L_i$, если такой существует. Тогда имея такой массив и индекс последнего элемента, можно идя по массиву $L$ восстановить искомую подпоследовательность.

Ясно, что для префикса длины 1 такой массив $L$ существует. А для поддержания массива $L$ достаточно при обновлении элемента $d_i$ элементу $L$ с индексом элемента $d_i$ присваивать индекс элемента $d_{i-1}$, если такой сущестует. Действительно, если инвариант для $L$ выполнялся, то он сохранится при переходе - если $d_i$ новый конечный элемент подпоследовательности максимальной длины, то перейдя от $d_i$ к цепочке индексов в $L$, восстанавливающей предыдущую максимальную подпоследовательность, получим ответ. Если же максимальная длина больше $i$, то цепочку, восстанавливающую ответ, не изменили.

\end{document}
