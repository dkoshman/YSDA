\documentclass{article}

\usepackage{cmap}  % should be before fontenc
\usepackage[T2A]{fontenc}
\usepackage[utf8]{inputenc}
\usepackage[russian]{babel}
\usepackage{amsmath,amssymb,amsthm}
\usepackage[pdftex,colorlinks=true,linkcolor=blue,urlcolor=red,unicode=true,hyperfootnotes=false,bookmarksnumbered]{hyperref}
\usepackage{indentfirst}

% так можно определять команду для повторяющихся обозначений,
% чтобы не набирать каждый раз заново
\newcommand{\E}{\ensuremath{\mathsf{E}}}  % матожидание
\newcommand{\D}{\ensuremath{\mathsf{D}}}  % дисперсия
\newcommand{\Prb}{\ensuremath{\mathsf{P}}}  % вероятностная мера

\newcommand{\eps}{\varepsilon}  % нормальная буква эпсилон
\renewcommand{\phi}{\varphi}  % нормальная буква фи

\renewcommand{\le}{\leqslant}  % нормальный знак <=
\renewcommand{\leq}{\leqslant}  % нормальный знак <=
\renewcommand{\ge}{\geqslant}  % нормальный знак >=
\renewcommand{\geq}{\geqslant}  % нормальный знак >=

\newtheorem{lemma}{Лемма}  % создаёт команд для лемм, можно сделать так же для любого другого вида утверждений

\pagestyle{myheadings}
\markright{Дмитрий Кошман\hfill}  % <- здесь нужно подставить свои имя и фамилию

\begin{document}

\section*{Теория к задаче Нечеткий поиск}
\section{Алгоритм решения}

Получаем на вход две строки: $pattern$ и $text$.
\smallskip

Если размер $pattern$ больше $text$ или одна из строчек пустая, возвращаем пустой массив.
\smallskip

Пусть символы $wildcard$ встречаются в строке $pattern$ на позициях $wildcard\_pos_i$, $i=1 \dots \#wildcards$. Тогда разбиваем $pattern$ по $wildcard$ символам на подстрочки, и составляем из них массив
$subpatterns$, $subpatterns[i] = pattern[begin_i: begin_{i+1} - 1)$, где $begin_0 = 0, begin_{i} = wildcard\_pos_i, begin_{\#wildcards + 1} = pattern.size + 1$.
\smallskip

Заводим массив $candidates$ размера $text.size$, в $i$-ой ячейке которого лежит массив булевых флагов размера $subpatterns.size = \#wildcards + 1$.
\smallskip

Запускаем алгоритм Ахо-Корасик, ищя подстрочки $subpatterns$ в $text$.
По мере увеличения размера обработанного префикса $text[0:i]$, алгоритм Ахо-Корасик будет возвращать нам множество индексов $\{j\}$, соответствующих строкам массива $subpatterns$, которые встречаются в тексте и заканчиваются на позиции $i$. Тогда, начиная с позиции $i = pattern.size$, мы выставляем соответвующие флаги $candidates[k_j][j]$, где $k_j$ - такая позиция в $text$, что если бы там начинался $pattern$, то мы бы увидели строчку $subpattern_j$  в этом же месте, то есть $k_j = i - pattern.size + begin_j + 1$.
\smallskip

После этого проходим по массиву $candidates$, и выписываем все индексы $i$, такие, что все $\#wildcards + 1$ флагов $candidates[i]$ отмечены.

\section{Доказательство правильности алгоритма}

Если размер $pattern$ больше $text$ или одна из строчек пустая, то вхождений быть не может.
\smallskip

Иначе если в $text$ есть вхождение $pattern$ на месте $i$, то алгоритм его найдет. Если бы $i$ не было выписано, то в $candidates[i]$ не отмечен какой-то флаг. Но это значит, что на ожидаемом месте не нашлась подстрока без $wildcard$ символов, что противоречит корректности алгоритма Ахо-Корасик.
\smallskip

И обратно, если алгоритм выдал индекс $i$, то там вхождение действительно есть, поскольку это означает, что все флаги в $candidates[i]$ отмечены, и все символы в $text[i:i + pattern.size)$, отличные от $wildcard$, совпадают с соответсвующими в $pattern$. 

\section{Временная сложность — асимптотика}

Нахождение позиций $wildcard\_pos_i$ и построение массива $subpatterns$  - $O(pattern)$.
\smallskip

Создание массива $candidates$ - $O(text * \# wildcards)$
\smallskip

Алгоритм Ахо-Корасик и заполнение массива $candidates$ - $O(text + pattern + \#occurrences) = O(text * \#wildcards)$.
\smallskip

Проход по массиву $candidates$ и выписывание ответов - $O(text)$.
\smallskip

\textbf{Общая сложность - $O(text * \#wildcards)$}.

\section{Затраты памяти — асимптотика}

Для массива $subpatterns$ -- $O(\# wildcards)$
\smallskip

Для массива $candidates$ - $O(text * \#wildcards)$
\smallskip

Для алгоритма Ахо-Корасик - $O(pattern)$
\smallskip

Для ответа - $O(text)$
\smallskip

\textbf{Общие затраты - $O(text * \#wildcards)$}.

\end{document}
