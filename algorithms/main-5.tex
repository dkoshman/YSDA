\documentclass{article}

\usepackage{cmap}  % should be before fontenc
\usepackage[T2A]{fontenc}
\usepackage[utf8]{inputenc}
\usepackage[russian]{babel}
\usepackage{amsmath,amssymb,amsthm}
\usepackage[pdftex,colorlinks=true,linkcolor=blue,urlcolor=red,unicode=true,hyperfootnotes=false,bookmarksnumbered]{hyperref}
\usepackage{indentfirst}

% так можно определять команду для повторяющихся обозначений,
% чтобы не набирать каждый раз заново
\newcommand{\E}{\ensuremath{\mathsf{E}}}  % матожидание
\newcommand{\D}{\ensuremath{\mathsf{D}}}  % дисперсия
\newcommand{\Prb}{\ensuremath{\mathsf{P}}}  % вероятностная мера

\newcommand{\eps}{\varepsilon}  % нормальная буква эпсилон
\renewcommand{\phi}{\varphi}  % нормальная буква фи

\renewcommand{\le}{\leqslant}  % нормальный знак <=
\renewcommand{\leq}{\leqslant}  % нормальный знак <=
\renewcommand{\ge}{\geqslant}  % нормальный знак >=
\renewcommand{\geq}{\geqslant}  % нормальный знак >=

\newtheorem{lemma}{Лемма}  % создаёт команд для лемм, можно сделать так же для любого другого вида утверждений

\pagestyle{myheadings}
\markright{Дмитрий Кошман\hfill}  % <- здесь нужно подставить свои имя и фамилию

\begin{document}

\section{Задача 3-1}

Рассмотрим аммортизированную стоимость операции $zig$:

$$cost' = cost + \Delta \Phi = 1 + \Phi'(x) + \Phi'(y) - \Phi(x) - \Phi(y)=$$
$$ = 1 + \Phi'(y) - \Phi(x) < 1 + \Phi'(x) - \Phi(x) < 1 + 3(\Phi'(x) - \Phi(x))$$

Аммортизированная стоимость операции $zig-zig$:

$$cost' = cost + \Delta \Phi = 1 + \Phi'(x) + \Phi'(y) + \Phi'(z) - \Phi(x) - \Phi(y) - \Phi(z)=$$
$$ =1 + \Phi'(y) + \Phi'(z) - \Phi(x) - \Phi(y)$$

Поскольку логарифм - выпуклая функция, то $\log\frac{a+b}{2} \geq \frac{\log a + \log b}{2} $, или $1 \leq \log (a+b) - \frac{\log a + \log b}{2}$. Применимо к данной функции потенциала с учетом расположения вершин, получаем $1 \leq \Phi'(x) - \frac{\Phi(x) + \Phi(z)}{2}$. Получаем:

$$cost' \leq \Phi'(x) - \frac{\Phi(x) + \Phi'(z)}{2} + \Phi'(y) + \Phi'(z) - \Phi(x) - \Phi(y) \leq $$

$$\leq \Phi'(x) - \frac{\Phi(x)}{2} + \Phi'(x) + \frac{\Phi'(z)}{2} - \Phi(x) - \Phi(x) \leq 3(\Phi'(x) - \Phi(x)) $$

Аналогичные рассуждения для $zig-zag$. Далее амортизационная оценка операции $splay$ получается аналогично случаю с функцией потенциала, зависящей от мощности поддерева.

\end{document}
